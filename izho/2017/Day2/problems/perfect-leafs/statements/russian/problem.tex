\begin{problem}{Крутой маршрут}{road.in}{road.out}{1 секунда}{256 мегабайт}

Мансур~--- является правителем страны АСМстан. В этой стране $N$ городов и $N-1$ двусторонних дорог. Известно что, с каждого города можно добраться до любого другого двигаясь по существующим дорогам. Более формально: страна выглядит как дерево, где вершинами являются города, а ребра двусторонние дороги.

Также, в этой стране города из которых выходит \textbf{ровно одна} дорога называются \textit{конечными}. Маршрутом называется простой путь от одного \textit{конечного} города до другого \textit{конечного} города. Расстояние между двумя городами есть минимальное число дорог которое требуется пройти чтобы дойти от одного города до другого. Расстоянием от какого-либо города до маршрута есть минимальное количество дорог, которое требуется пройти от заданного города до какого-либо города на маршруте. Мансур решил внедрить в город \textbf{ровно один} маршрут, но Мансур заинтересован только в \textit{крутых} маршрутах. Крутость маршрута считается следующим способом: пусть $A$ и $B$ это \textit{конечные} города данного маршрута, а $H$ это максимальное расстояние от какого-либо города в стране до этого маршрута, тогда \textit{крутость} данного маршрута это произведение расстояния между $A$ и $B$ на $H$.

Мансур дал задание Темирулану найти максимальную \textit{крутость} среди всех маршрутов, более того ему интересно знать количество таких маршрутов. Темирулан просит помощи у Вас. 

Настоятельно рекомендуем прочесть пояснение к примеру.

\InputFile
В первой строке входных данных содержится целое положительное число $N$ ($2 \le N \le 500000$)~--- количество городов в стране. Города пронумерованы от $1$ до $N$. В следующих $N-1$ строках содержится по $2$ целых положительных числа, разделенных пробелом, $u_i$, $v_i$ ($1 \le u_i, v_i \le N; u_i \neq v_i$)~--- дорога соединяющая города $u_i$ и $v_i$. Гарантируется, что заданный граф дерево.

\OutputFile
В единственной строке выходных данных выведите два целых числа~--- максимальную \textit{крутость} и количество маршрутов, разделенные пробелом. \textbf{Обратите внимание}, что маршрут от $A$ до $B$ и маршрут от $B$ до $A$ считаются \textbf{одним и тем же} маршрутом.

\Scoring
Данная задача содержит три подзадачи:
\begin{enumerate}
\item $2 \le N \le 100$. Оценивается в $19$ баллов.
\item $2 \le N \le 5000$. Оценивается в $33$ баллов.
\item $2 \le N \le 500000$. Оценивается в $48$ баллов.
\end{enumerate}

Каждая подзадача оценивается только при прохождении всех предыдущих.

\Examples

\begin{example}
\exmp{7
1 2
1 3
2 4
2 5
3 6
3 7
}{6 2
}%
\exmp{4
1 2
2 3
2 4
}{2 3
}%
\exmp{5
1 2
2 3
3 4
4 5
}{0 1
}%
\end{example}

\Note
Путь называется простым, если вершины в нем не повторяются. Обратите внимание, что не все простые пути являются \textit{маршрутом}.

В первом тестовом примере:

Четыре \textit{конечных} города с номерами 4, 5, 6 и 7. Если выбрать маршрутом путь 4-2-1-3-6, то расстояние между ними равно 4 и расстояния от других городов до этого маршрута равны [1, 1], максимальное среди них 1, соответственно \textit{крутость} маршрута равна $4 \times 1 = 4$. А если выбрать маршрутом путь 4-2-5, то расстояние равно 2, а максимальное расстояние до этого маршрута равно 3, \textit{крутость} маршрута равна $3 \times 2 = 6$. \textit{Крутость} маршрута 6-3-7 также равна 6, остальные маршруты имеют меньшую \textit{крутость}.

В третьем тестовом примере есть только два \textit{конечных} города 1 и 5, поэтому есть ровно один маршрут 1-2-3-4-5, расстояние равно 4, а максимальное расстояние до этого маршрута равно 0, т.к. все города лежат на этом маршруте. Соответственно \textit{крутость} равна $4 \times 0 = 0$.

\end{problem}

