\begin{problem}{Simple game}{game.in}{game.out}{1 second}{256 megabytes}

NurlashKO was well behaved during last year, for this Ded Moroz gifted him for New Year polygonal chain line with $N$ vertices. $i$-th vertex of this chain located at the point with coordinates $(i, y_i)$.

Very soon a new game with this geometric figure was invented: the following operations are executed $M$ times: 
\begin{itemize}  
   \item Change $y$ coordinate for one of the chain vertexes.
   \item Draw a horizontal line at the height $H$ and count its intersections with the chain. Note, that all points of horizontal line have $y$ coordinate equal to $H$.
\end{itemize}

NurlashKO likes this game and he asks your help to write a program for this game.


\InputFile
First line of input contains two positive integers $N, M (1 \leq N, M \leq 100\,000)$~--- the numbers of vertices and operations in the game, respectively.

Next line contains $N$ positive integers separated by a single space $h_i (1 \leq h_i \leq 1\,000\,000)$~--- $h_i$ is the original height of the $i$-th vertex.

Then in $M$ lines follows descriptions of the game operations in the following format:
\begin{itemize}
\item $1$ $pos$ $val$ $(1 \leq pos \leq N, 1 \leq val \leq 1\,000\,000)$~--- vertex number and its new new height, respectively.
\item $2$ $H$ $(1 \leq H \leq 1\,000\,000)$~--- height of the horizontal line. It is guaranteed that this line will never intersects with the chain at the vertices.
\end{itemize}

\OutputFile
For each query of the second type on a separate line output the number of intersections of horizontal line with the chain. Output answers to queries in the same order as they appear in the input file.


\Scoring
This problem consists of 3 subtasks:
\begin{enumerate}
\item $1 \leq N, M \leq 1\,000$. Score $22$ points.
\item $1 \leq N, M \leq 100\,000$. Only query (second type) operations are allowed. Score $27$ points.
\item $1 \leq N, M \leq 100\,000$. Score $51$ points.
\end{enumerate}

Each subtask will be scored only if the solution successfully passes all of the previous subtasks.

\Example

\begin{example}
\exmp{3 3
1 5 1
2 3
1 1 5
2 3
}{2
1
}%
\end{example}

\end{problem}

